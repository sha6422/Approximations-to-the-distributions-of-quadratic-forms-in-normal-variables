\documentclass[12pt]{beamer}
\usetheme{CambridgeUS}%{Berkeley}%
\mode<presentation>
%\usepackage{beamerthemesplit}
\usepackage{amsmath,amssymb,latexsym,epsfig,graphics}
\linespread{2}
\title{Generalized Linear, Non-linear and Mixed Models}
\subtitle{Theory, Methods and Practice}
%\author{}
\date{\today}

\begin{document}

\frame{\titlepage}

\section*{Outline}
\frame{\tableofcontents }



\section{Introduction and Motivation}

\begin{frame}
\begin{itemize}
  \item<1-> What is regression? What does regression do?
  \item<2-> Regression ingredient:
    \begin{enumerate}
    \item<1-> Response or dependent variable, usually denoted by
    $Y$; $y$
    \item<2-> Auxiliary or independent variable, usually denoted by
    $X$; $x$
    \end{enumerate}
  \item<3-> Both $Y$ and $X$ may be univariable or multivariable.
  \item<4-> Dual Purpose:
    \begin{enumerate}
    \item<1-> Develop relationship between $Y$ and $X$;
    \item<2-> Predict future or unobserved $Y$ using known or
    observed $X$.
    \end{enumerate}
  \end{itemize}
\end{frame}

\begin{frame}
  \begin{itemize}
  \item<1-> Deterministic world: $Y=f(x)$, $f$ is completely known
  function.\\
   e.g. $A=\pi r^2$, $Y=A$, $x=r$, $f=\pi r^2$.
   \item<2-> Indeterministic world or uncertain events\\
   $Y\approx (x)$\\
   or, $Y=f(x)+\varepsilon$.
  \item<3-> Two components:\\
  $f$: Mathematical world;\\
  $\varepsilon$: Statistical world.
  \end{itemize}
  \end{frame}


\begin{frame}
  \begin{itemize}
 \item<1-> However, $f$ and $\varepsilon$ could be related (\alert{
 How?}).

 \item<2->Not necessarily statistically rather in terms of
 mathematical measurements

 \item<3-> Linear Regression:
    $Y=\beta_{0}+\beta_{1}x+\varepsilon$.
    \item<4-> What are the basic assumptions made?

    \begin{enumerate}
      \item<1-> For every given $x$, $Y$ is a random variable.
      \item<2-> The mean of $Y$ is linearly related to $x$ and in
      particular it is a straight line equation.
  \end{enumerate}
   \end{itemize}
\end{frame}

\begin{frame}

      Other usual assumptions:
      \begin{enumerate}
      \item<1-> $\varepsilon$'s are independent and identically
      distributed (iid).
      \item<2-> $\varepsilon$'s are normally distributed: why this
      is required? In an indeterministic world measurement
      of uncertainty is the main thing and the uncertainty
      is measured by probability. The normality assumptions
      provide accurate calculation of uncertainty associated
      the $Y$ and $x$ relationship.
    \end{enumerate}
\end{frame}


\begin{frame}
  \begin{itemize}
    \item<1-> Interpretation of $\beta$'s: Important for many
    practical applications.

    \item<2-> How do you determine $\beta$? Why do they need to be
    fixed?

    \item <3-> For a given data set many straight lines seem
    plausible but we want one that uniquely determine the
    relationship. The uniqueness is defined by minimum squared
    error loss, in mathematical sense.
    \end{itemize}
  \end{frame}
\begin{frame}
  \begin{itemize}
    \item<1->However, they are uniquely determined by maximum
    likelihood estimation or some other methods (will be learned
    in this course) in statistical sense. Although they could be
    identical in many special cases.
    \item<2-> Minimum Distance method
     \begin{equation}
     min_{\beta} \sum_{i=1}^{n}(y_{i}-\beta_{0}-\beta_{x}x_{i}^2)
     \end{equation}
   \end{itemize}
\end{frame}

\begin{frame}
  \begin{itemize}
     \item<1-> ML method
     \begin{equation}
     max_{\beta}logL(\beta)= -\frac{n}{2}log(2\pi)-\frac{n}{2}log\sigma^{2}-\frac{1}{2\sigma^2} \sum_{i=1}^{n}(y_{i}-\beta_{0}-\beta_{2}x_{i})^2
     \end{equation}
     \item<2-> You are required to know the statistical properties of
  the estimators of $\beta$
  \end{itemize}
\end{frame}


%\begin{frame}
%  \begin{itemize}
%    \item<1->
%    \item<2->
%  \end{itemize}
%\end{frame}

\section{Introduction to Logistic Regression and GLM}

\begin{frame}

  \begin{itemize}
    \item<1-> Consider the data set extracted from a national study
    of $15$ and $16$ year old adolescents. The event of interest
    is ever having sexual intercourse.
  \end{itemize}

\begin{center}
  \fontsize{10}{11}\selectfont{\begin{tabular}{cccc}\hline
    Race&Gender& \multicolumn{2}{c}{Intercourse}\\
    &&Yes&No\\
    \hline
    White & Male & 43 & 134\\
    & Female & 26 & 149\\
    \hline
    Black & Male & 29 & 23\\
    & Female & 22 & 36\\
    \hline
    \end{tabular}}
\end{center}
\end{frame}


\begin{frame}
  \begin{itemize}
    \item<1->
    \begin{equation}
    \begin{split}
    \pi&=Pr(intercourse)\\
    &=\frac{43+26+29+22}{(43+26+29+22)+(134+149+23+36)}=0.260
    \end{split}
    \end{equation}
    \item<2->
    \begin{equation}
    Pr(intercourse|white)=\frac{43+26}{43+26+134+149}=0.196
    \end{equation}
  \end{itemize}
\end{frame}

\begin{frame}
  \begin{itemize}
    \item<1->
    \begin{equation}
    Pr(intercourse|Black)=\frac{29+22}{29+22+23+36}=0.463
    \end{equation}
    \item<2->
    \begin{equation}
    Pr(intercourse|Male)=\frac{43+29}{43+29+134+23}=0.314
    \end{equation}
    \item<3->
    \begin{equation}
    Pr(intercourse|Female)=\frac{26+22}{26+22+149+36}=0.206
    \end{equation}
  \end{itemize}
\end{frame}

\begin{frame}
  \begin{itemize}
    \item<1->
    \begin{equation}
    Pr(intercourse|White \& Male)=\frac{43}{43+134}=0.243
    \end{equation}
    \item<2->
    \begin{equation}
    Pr(intercourse|White \& Female)=\frac{26}{26+149}=0.148
    \end{equation}
  \end{itemize}
\end{frame}
\begin{frame}
  \begin{itemize}
    \item<1->
    \begin{equation}
    Pr(intercourse|Black \& Male)=\frac{29}{29+23}=0.558
    \end{equation}
    \item<2->
    \begin{equation}
    Pr(intercourse|Black \& Female)=\frac{22}{22+36}=0.379
    \end{equation}
  \end{itemize}
\end{frame}



\begin{frame}
  \begin{itemize}
    \item<1-> How about having a formula: $\pi (x)=f(x)$?

    what $x$?
    \item<2-> Define conveniently
    \begin{equation*}
     x_{2}=\left\{
      \begin{array}{rl}
      1  &  \text{if Male}\\
      0  &  \text{if Female}
      \end{array} \right.
    \end{equation*}

  \begin{equation*}
     x_{1}=\left\{
      \begin{array}{rl}
      1  &  \text{if White}\\
      0  &  \text{if Black}
      \end{array} \right.
    \end{equation*}
  \end{itemize}
\end{frame}


\begin{frame}
  \begin{itemize}
    \item<1-> Let
    $f(x)=\beta_{0}+\beta_{1}x_{1}+\beta_{2}x_{2}=-0.455-0.1313x_{1}+0.648x_{2}$.

    \begin{center}
      \begin{tabular}{cccl} \hline
      \multicolumn{2}{c}{X}& $\pi$(x)&\\
      1&1&0.329&White Male\\
      1&0&0.204&White Female\\
      0&1&0.646&Black Male\\
      0&0&0.488&Black Female\\
      \hline
      \end{tabular}
    \end{center}
  \end{itemize}
\end{frame}


\begin{frame}
  Issues:
  \begin{itemize}
    \item<1-> Structural defect-Probabilities fall between $0$ and
    $1$, where as linear functions take values over the entire
    real line. The model can be valid for a specific range of
    $x$, but not for all values of $x$.
  \end{itemize}
  Require a more general model formation.
\end{frame}



\end{document}
